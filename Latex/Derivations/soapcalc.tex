\title{\LaTeXe{} Physics Exercises}
\author{
        Eiaki Morooka (Aki) \\
                Department of IT Engineering\\
        Metropolia University of Applied Sciences\\
        Karaportti 2, 02610 Espoo, \underline{Finland}
}
\date{\today}

\documentclass[10pt]{article}
\usepackage{amsmath}
\usepackage{amssymb}
%\usepackage{a4wide}
\usepackage[top=30pt,bottom=40pt,left=40pt,right=40pt]{geometry}
\allowdisplaybreaks

\begin{document}
\maketitle

\section*{SOAP Spectrum}

\begin{align}
c^i_{nlm} &= \int dV g_{nl} (r_i) \rho(\vec{r}_i) Y_{lm}(\theta, \phi) \\
          &= \int dV \sum_k \beta_{nk} r^l e^{-\alpha_{kl} r_i^2}  e^{(\vec{r} - \vec{r_i})^2} Y_{lm}(\theta, \phi) \\
          &= \int dV \sum_k \beta_{nk} e^{-\alpha_{kl} r_i^2}  e^{(\vec{r} - \vec{r_i})^2} \Phi_{lm}(r, \theta, \phi) \\
          &= \int dV \sum_k \beta_{nk} e^{- (1+\alpha_{kl})(\vec{r} - \frac{\vec{r_i}}{1+\alpha_{kl}})^2} e^{-\frac{\alpha_{kl} }{1+\alpha_{kl}} r^2_i} \Phi_{lm}(x, y, z) \\
          &=  \sum_k \beta_{nk} e^{-\frac{\alpha_{kl}}{1+\alpha_{kl}} r^2_i} \int dV  e^{- (1+\alpha_{kl})(\vec{r} - \frac{\vec{r_i}}{1+\alpha_{kl}})^2}  \Phi_{lm}(x, y, z) \\
          &=  \sum_k \beta_{nk} e^{-\frac{\alpha_{kl}}{1+\alpha_{kl}} r^2_i} \int dV  e^{- (1+\alpha_{kl})\vec{r} ^2}  \Phi_{lm}(x+\frac{x_i}{1+\alpha_{kl}}, y +\frac{y_i}{1+\alpha_{kl}}, z+\frac{z_i}{1+\alpha_{kl}}) \label{infinity} \\
          &=  \sum_k \beta_{nk} e^{-\frac{\alpha_{kl}}{1+\alpha_{kl}} r^2_i} \Phi_{lm}(\frac{x_i}{1+\alpha_{kl}}, \frac{y_i}{1+\alpha_{kl}}, \frac{z_i}{1+\alpha_{kl}}) \int dV  e^{- (1+\alpha_{kl})\vec{r} ^2}   \\
          &=  \frac{\pi^{3/2}}{(1+\alpha_{lm})^{3/2}}\sum_k \beta_{nk} e^{-\frac{\alpha_{kl}}{1+\alpha_{kl}} r^2_i} \Phi_{lm}(\frac{x_i}{1+\alpha_{kl}}, \frac{y_i}{1+\alpha_{kl}}, \frac{z_i}{1+\alpha_{kl}}) \\
          &=  \frac{\pi^{3/2}}{(1+\alpha_{lm})^{3/2 + l}}\sum_k \beta_{nk} e^{-\frac{\alpha_{kl}}{1+\alpha_{kl}} r^2_i} P_{lm}(x_i, y_i, z_i)  \label{harpoly}
\end{align}

eq.\ref{infinity} is possible because x,y and z go to plus and minus infinity, and because it is in the form of a polynomial. And eq.\ref{harpoly} comes from the characteristics of the spherical harmonics,
\begin{align}
\Phi_{lm}(\frac{x}{a},\frac{y}{a},\frac{z}{a}) = r^l Y_{lm}(\frac{x}{a},\frac{y}{a},\frac{z}{a})  = \frac{1}{a^l} P_{lm}(x,y,z)
\end{align}
where $a^l$ is a constant and $P_{lm}$ takes a polynomial form. Hence the derivatives are
\begin{align}
         \frac{\partial c^i_{nlm}}{ \partial x_i} &=  \frac{\xi_l\pi^{3/2}}{(1+\alpha_{lm})^{3/2 + l}}\sum_k \beta_{nk} \left( \frac{\partial P_{lm}(x_i, y_i, z_i)}{\partial x_i} e^{-\frac{\alpha_{kl}}{1+\alpha_{kl}} r^2_i} - \frac{2\alpha_{kl}x_i}{1+\alpha_{kl} }e^{-\frac{\alpha_{kl}}{1+\alpha_{kl}} r^2_i} P_{lm}(x_i, y_i, z_i)  \right) \\
         \frac{\partial c^i_{nlm}}{ \partial y_i} &=  \frac{\xi_l\pi^{3/2}}{(1+\alpha_{lm})^{3/2 + l}}\sum_k \beta_{nk} \left( \frac{\partial P_{lm}(x_i, y_i, z_i)}{\partial y_i} e^{-\frac{\alpha_{kl}}{1+\alpha_{kl}} r^2_i} - \frac{2\alpha_{kl}y_i}{1+\alpha_{kl} }e^{-\frac{\alpha_{kl}}{1+\alpha_{kl}} r^2_i} P_{lm}(x_i, y_i, z_i)  \right) \\
         \frac{\partial c^i_{nlm}}{ \partial z_i} &=  \frac{\xi_l\pi^{3/2}}{(1+\alpha_{lm})^{3/2 + l}}\sum_k \beta_{nk} \left( \frac{\partial P_{lm}(x_i, y_i, z_i)}{\partial z_i} e^{-\frac{\alpha_{kl}}{1+\alpha_{kl}} r^2_i} - \frac{2\alpha_{kl}z_i}{1+\alpha_{kl} }e^{-\frac{\alpha_{kl}}{1+\alpha_{kl}} r^2_i} P_{lm}(x_i, y_i, z_i)  \right)
\end{align}

\end{document}
